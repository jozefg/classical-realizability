\documentclass{amsart}
\usepackage{amsmath, amssymb, amsthm}
\usepackage[parfill]{parskip}

\title{Classical Realizability}
\author{Danny Gratzer}
\begin{document}
\maketitle

\section{An Overview of Realizability}

%% TODO: This should be less chummy.

In this note I'd like to talk about a realizability model for
classical logic, but it's worth writing down a few thoughts on how
realizability in general works to make sure that we're on the same
page.

Realizability is a technique for understanding a logic from a
computational point of view. Rather than studying the proofs in a
logic (proof theory) we define a system of computation and model
propositions using collections of terms in that system. This technique
dates back to Kleene's model of intuitionistic logic. The primary
benefit of a realizability treatment is that it's an explanation of
what the computational content of a theorem is. One interesting
consequence for doing this for classical logic is to see that a
computational interpretations aren't limited to constructive logics
(just nice ones).

\section{A Judgmental Formulation of Classical Logic}

\section{A Computation System}

\section{The Realizability Model}

\section{Soundness and Completeness}

\end{document}
