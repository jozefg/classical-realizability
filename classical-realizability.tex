\documentclass{amsart}
\usepackage{amsmath, amssymb, amsthm}
\usepackage[parfill]{parskip}

\usepackage{mathpartir}

%% mathbf looks funky but I like it better than mathtt,
%% alternatives?
\newcommand{\true}[1]{\ensuremath{#1\ \mathbf{true}}}
\newcommand{\false}[1]{\ensuremath{#1\ \mathbf{false}}}
\newcommand{\contra}{\ensuremath{\mathbf{contra}}}
\newcommand{\obvious}[1]{\inferrule{ }{#1}}

\title{Classical Realizability}
\author{Danny Gratzer}
\begin{document}
\maketitle

\section{An Overview of Realizability}

%% TODO: This should be less chummy.

In this note I'd like to talk about a realizability model for
classical logic, but it's worth writing down a few thoughts on how
realizability in general works to make sure that we're on the same
page.

Realizability is a technique for understanding a logic from a
computational point of view. Rather than studying the proofs in a
logic (proof theory) we define a system of computation and model
propositions using collections of terms in that system. This technique
dates back to Kleene's model of intuitionistic logic. The primary
benefit of a realizability treatment is that it's an explanation of
what the computational content of a theorem is. One interesting
consequence for doing this for classical logic is to see that a
computational interpretations aren't limited to constructive logics
(just nice ones).

\section{A Judgmental Formulation of Classical Logic}

In order to formalize a computational interpretation of classical
logic, we first have to formally define classical logic. I'll do this
with the normal proof theoretic tools for doing this. We start by
fixing some set of propositions we're interested in studying.

\[
  %% TODO: Generalize with \forall and \exists for a more interesting
  %% model?
  \begin{array}{lcl}
    P & ::= & \top \mid \bot \mid P \wedge P \mid P \vee P \mid P
              \supset P \mid \neg P
  \end{array}
\]

We'd like to formulate the judgment \true{P}. Unlike in an
intuitionistic logic, classical logic also needs the judgment
\false{P} since proofs in this logic switch freely between searching
for contradictions and building evidence. We also have a judgment for
denoting a contradiction, \contra. This gives us some way to mediate
between \true{P} and \false{P} and also let's us state double
negation in a cleaner way.

%% A constructive logic would never be so duplicated, my sense of smug
%% superiority can be left intact.
\begin{mathpar}
  \inferrule{ }{\true{\top}} \and
  \inferrule{ }{\false{\bot}} \and
  \inferrule{\true{A} \\ \true{B}}{\true{A \wedge B}} \and
  \inferrule{\false{A}}{\false{A \wedge B}} \and
  \inferrule{\false{B}}{\false{A \wedge B}} \and
  \inferrule{\true{A}}{\true{A \wedge B}} \and
  \inferrule{\true{B}}{\true{A \wedge B}} \and
  \inferrule{\false{A} \\ \false{B}}{\false{A \wedge B}} \and
  \inferrule{\true{B}\ (\true{A})}{\true{A \supset B}} \and
  \inferrule{\false{B}\ (\true{A})}{\false{A \supset B}} \and
  \inferrule{\true{A}}{\false{\neg A}} \and
  \inferrule{\false{A}}{\true{\neg A}}
\end{mathpar}

These rules govern the connective specific aspects of $\true{P}$ and
$\false{P}$ but the real essence of the classical logic comes from how
we let these two judgments interact through \contra,

\begin{mathpar}
  \inferrule{\true{A} \\ \false{A}}{\contra} \and
  \inferrule{\contra \ (\false{A})}{\true{A}} \and
  \inferrule{\contra \ (\true{A})}{\false{A}}
\end{mathpar}

Notice that I haven't defined any elimination rules for this
system. It turns out that we can derive all of the by heavy use of
contradiction.

\[
  \inferrule
  {\inferrule{\inferrule{\obvious{\false{A}\ (\false{A})}}
                        {\false{A \wedge B} \ (\false{A})} \\
              \inferrule{D}{\true{A \wedge B}}}
                        {\contra\ (\false{A})}}
  {\true{A}}
\]

Now the middle rule governing contradiction gives us double negation
easily enough

\section{A Computation System}

\section{The Realizability Model}

\section{Soundness and Completeness}

\end{document}
