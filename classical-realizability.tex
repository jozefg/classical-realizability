\documentclass{amsart}
\usepackage{amsmath, amssymb, amsthm}
\usepackage[parfill]{parskip}

\usepackage{mathpartir}

%% mathbf looks funky but I like it better than mathtt,
%% alternatives?
\newcommand{\true}[1]{\ensuremath{#1\ \mathbf{true}}}
\newcommand{\false}[1]{\ensuremath{#1\ \mathbf{false}}}
\newcommand{\contra}{\ensuremath{\mathbf{contra}}}
\newcommand{\obvious}[1]{\inferrule{ }{#1}}

\newcommand{\val}[1]{#1\ \mathbf{val}}
\newcommand{\push}{\rhd}
\newcommand{\pull}{\lhd}
\newcommand{\ap}[1]{\mathtt{ap}(-;\ #1)}
\newcommand{\throw}[1]{\mathtt{throw}(#1;\ -)}
\newcommand{\callcc}{\mathtt{callcc}~}
\newcommand{\cont}{\mathtt{cont}}
\newcommand{\ite}[3]{\mathtt{if}\ #1\ \mathtt{then}\ #2\ \mathtt{else}\ #3}

\title{Classical Realizability}
\author{Danny Gratzer}
\begin{document}
\maketitle

\section{An Overview of Realizability}

%% TODO: This should be less chummy.

In this note I'd like to talk about a realizability model for
classical logic, but it's worth writing down a few thoughts on how
realizability in general works to make sure that we're on the same
page.

Realizability is a technique for understanding a logic from a
computational point of view. Rather than studying the proofs in a
logic (proof theory) we define a system of computation and model
propositions using collections of terms in that system. This technique
dates back to Kleene's model of intuitionistic logic. The primary
benefit of a realizability treatment is that it's an explanation of
what the computational content of a theorem is. One interesting
consequence for doing this for classical logic is to see that a
computational interpretations aren't limited to constructive logics
(just nice ones).

\section{A Judgmental Formulation of Classical Logic}

In order to formalize a computational interpretation of classical
logic, we first have to formally define classical logic. I'll do this
with the normal proof theoretic tools for doing this. We start by
fixing some set of propositions we're interested in studying.

\[
  %% TODO: Generalize with \forall and \exists for a more interesting
  %% model?
  \begin{array}{lcl}
    P & ::= & \top \mid \bot \mid P \wedge P \mid P \vee P \mid P
              \supset P \mid \neg P
  \end{array}
\]

We'd like to formulate the judgment \true{P}. Unlike in an
intuitionistic logic, classical logic also needs the judgment
\false{P} since proofs in this logic switch freely between searching
for contradictions and building evidence. We also have a judgment for
denoting a contradiction, \contra. This gives us some way to mediate
between \true{P} and \false{P} and also let's us state double
negation in a cleaner way.

%% A constructive logic would never be so duplicated, my sense of smug
%% superiority can be left intact.
\begin{mathpar}
  \inferrule{ }{\true{\top}} \and
  \inferrule{ }{\false{\bot}} \and
  \inferrule{\true{A} \\ \true{B}}{\true{A \wedge B}} \and
  \inferrule{\false{A}}{\false{A \wedge B}} \and
  \inferrule{\false{B}}{\false{A \wedge B}} \and
  \inferrule{\true{A}}{\true{A \vee B}} \and
  \inferrule{\true{B}}{\true{A \vee B}} \and
  \inferrule{\false{A} \\ \false{B}}{\false{A \vee B}} \and
  \inferrule{\true{B}\ (\true{A})}{\true{A \supset B}} \and
  \inferrule{\true{A} \\ \false{B}}{\false{A \supset B}} \and
  \inferrule{\true{A}}{\false{\neg A}} \and
  \inferrule{\false{A}}{\true{\neg A}}
\end{mathpar}

These rules govern the connective specific aspects of $\true{P}$ and
$\false{P}$ but the real essence of the classical logic comes from how
we let these two judgments interact through \contra,

\begin{mathpar}
  \inferrule{\true{A} \\ \false{A}}{\contra} \and
  \inferrule{\contra \ (\false{A})}{\true{A}} \and
  \inferrule{\contra \ (\true{A})}{\false{A}}
\end{mathpar}

Notice that I haven't defined any elimination rules for this
system. It turns out that we can derive all of the by heavy use of
contradiction.

\[
  \inferrule
  {\inferrule{\inferrule{\obvious{\false{A}\ (\false{A})}}
                        {\false{A \wedge B} \ (\false{A})} \\
              \inferrule{D}{\true{A \wedge B}}}
                        {\contra\ (\false{A})}}
  {\true{A}}
\]

Now the middle rule governing contradiction gives us double negation
easily enough

\[
  \inferrule
  {\inferrule
   {\inferrule
    {\inferrule
     {\obvious{\false{A \vee \neg A}\ (\false{A \vee \neg A})}}
     {\false{A}\ (\false{A \vee \neg A})}}
    {\true{\neg A}\ (\false{A \vee \neg A})}\\
    \inferrule
     {\obvious{\false{A \vee \neg A}\ (\false{A \vee \neg A})}}
     {\false{\neg A}\ (\false{A \vee \neg A})}}
   {\contra\ (\false{A \vee \neg A})}}
  {\true{A \vee \neg A}}
\]

The remainder of the elimination rules as well as the metatheoretic
properties of this system aren't going to be described.

\section{A Computation System}

Before we proceed with our realizability model we have to describe a
computation system to use to model our programming language. Normally
some variant of the lambda calculus suffices but for classical logic
it's more convenient if our computation system includes
continuations. As is typical with realizability, we
won't specify the entirety of the language. Instead we'll describe
several inferences we require to be valid in the system and leave
ourselves open to the possibility of more rules being added.

The trouble is that continuations are hard to give nice operational
semantics to. To handle this, I've opted to give a stack machine
presentation of the language. So our semantics will have three
judgments, $\val{e}$, $k \push e$ and $k \pull e$. In the second
judgment we're working by examining some term $e$ and building up some
surrounding context/continuation $k$. This judgment doesn't really do
any work, it sets us up for $k \pull e$ which takes some $e$ that we
don't know how to reduce any further and works back through $k$
performing reductions as it goes. For example, the inferences
governing how lambda abstraction and application in our language work
are

\begin{mathpar}
  \inferrule{ }{k \push e\ e' \mapsto k;\ \ap{e'} \push e} \and
  \inferrule{ }{k \push \lambda x.\ e \mapsto k \pull \lambda x. e} \and
  \inferrule{ }{k;\ \ap{e'} \pull \lambda x. e \mapsto k \push [e'/x]e}
\end{mathpar}

When evaluating an application we push a new frame onto our stack of
continuations and proceed in evaluation the function. Whenever we
reach a $\lambda$ (or any value as we'll see) we stop trying to use
$\push$ because we have no more expressions to decompose and we begin
to work on $k$. In particular, whenever we try to $\pull$ a lambda
through a continuation representing application we actually perform
the relevant substitution.

In addition to abstraction and application, we only need one more
construct in our language. This means the grammar of our language may
be specified as


\[
  \begin{array}{lcl}
    e & ::= & x \mid \lambda x. e \mid e\ e \mid \callcc x. e \mid
              \cont[k] \mid e \gg e \mid ...\\
    k & ::= & \ap{e} \mid \throw{e} \mid ...
  \end{array}
\]

The extra constructs we need specify how first class continuations
work in our language. The full set of inferences we demand in our
system is

\begin{mathpar}
  \inferrule{ }{\val{\lambda x. e}} \and
  \inferrule{ }{\val{\cont[k]}} \and
  \inferrule{\val{e}}{k \push e \mapsto k \pull e} \and
  \inferrule{ }{k \push e\ e' \mapsto k;\ \ap{e'} \push e} \and
  \inferrule{ }{k;\ \ap{e'} \pull \lambda x. e \mapsto k \push [e'/x]e} \and
  \inferrule{ }{k \push \callcc x. e \mapsto k \push [\cont[k]/x]e} \and
  \inferrule{ }{k \push e \gg e' \mapsto k; \throw{e} \push e'} \and
  \inferrule{ }{k'; \throw{e} \pull \cont[k] \mapsto k \push e} \and
\end{mathpar}

Now we also want to have nice syntax for several derived forms.

\begin{align*}
  (e_1, e_2) &\triangleq \lambda f. f\ e_1\ e_2\\
  \pi_1\ e &\triangleq e\ (\lambda x. \lambda y. x)\\
  \pi_2\ e &\triangleq e\ (\lambda x. \lambda y. y)\\
  tt &\triangleq \lambda x. \lambda y. x\\
  ff &\triangleq \lambda x. \lambda y. y\\
  \ite{e}{e_1}{e_2} &\triangleq e\ e_1\ e_2
\end{align*}

As it turns out, this will be plenty to formalize our theory.

\section{The Realizability Model}

\section{Soundness and Completeness}

\end{document}
